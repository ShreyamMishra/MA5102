
\documentclass[a4paper,12pt]{article} % 
\usepackage[top = 2.5cm, bottom = 2.5cm, left = 2.5cm, right = 2.5cm]{geometry} 
\usepackage{defs}
\usepackage[T1]{fontenc}
\usepackage[utf8]{inputenc}
\usepackage{graphicx} 
\usepackage{mathtools}
\usepackage{tikz,lipsum,lmodern}
\usepackage[most]{tcolorbox}
\usepackage{setspace}
\setlength{\parindent}{0in}
\usepackage{float}
\usepackage{fancyhdr}
\usepackage{lastpage}
% Installing necessary math packages
\usepackage{amsmath}
\usepackage{amssymb}
\usepackage{latexsym}
%%%%%%%%%%%%%%%%%%%%%%%%%%%%%%%%%%%%%%%%%%%%%%%%
% 3. Header (and Footer)
%%%%%%%%%%%%%%%%%%%%%%%%%%%%%%%%%%%%%%%%%%%%%%%%

% To make our document nice we want a header and number the pages in the footer.

\pagestyle{fancy} 
\fancyhf{}

\lhead{Assignment}% \lhead puts text in the top left corner. \footnotesize sets our font to a smaller size.

%\rhead works just like \lhead (you can also use \chead)
\rhead{Shreyam Mishra, 19D110020} %<---- Fill in your details.
% Similar commands work for the footer (\lfoot, \cfoot and \rfoot).
% We want to put our page number in the center.
\rfoot{Page \thepage \hspace{1pt} of \pageref{LastPage}}
\cfoot{\footnotesize \thepage} 


%%%%%%%%%%%%%%%%%%%%%%%%%%%%%%%%%%%%%%%%%%%%%%%%
% 4. Your document
%%%%%%%%%%%%%%%%%%%%%%%%%%%%%%%%%%%%%%%%%%%%%%%%

% Now, you need to tell LaTeX where your document starts. We do this with the \begin{document} command.
% Like brackets every \begin{} command needs a corresponding \end{} command. We come back to this later.

\begin{document}

%%%%%%%%%%%%%%%%%%%%%%%%%%%%%%%%%%%%%%%%%%%%%%%%
%%%%%%%%%%%%%%%%%%%%%%%%%%%%%%%%%%%%%%%%%%%%%%%%

%%%%%%%%%%%%%%%%%%%%%%%%%%%%%%%%%%%%%%%%%%%%%%%%
% Title section of the document
%%%%%%%%%%%%%%%%%%%%%%%%%%%%%%%%%%%%%%%%%%%%%%%%

% For the title section we want to reproduce the title section of the Problem Set and add your names.

\thispagestyle{empty} % This command disables the header on the first page. 


\begin{center} % Everything within the center environment is centered.
	{\Large \bf MA 5102 - Algebraic Topology}
	\vspace{2mm} % to add some vertical space in between the lines
	
	\bf{Assignment} % <---- Don't forget to put in the right number
	
	\vspace{2mm}
	
        % YOUR NAMES GO HERE
	{\bf Shreyam Mishra, 19D110020} % <---- Fill in your name and ID number here.
		
\end{center}  

\vspace*{0.3cm} 

\begin{center}
    \textbf{\large{Topological Properties of the Orthogonal Group}}
\end{center}
%%%%%%%%%%%%%%%%%%%%%%%%%%%%%%%%%%%%%%%%%%%%%%%%
%%%%%%%%%%%%%%%%%%%%%%%%%%%%%%%%%%%%%%%%%%%%%%%%

% Up until this point you only have to make minor changes for every assignment (Number of the assignment). Your write up essentially starts here.

%\textbf{Question 2:} \emph{Prove an analogue of Theorem 1.8 for $n$ functions}. \\
%\textbf{Solution:} LSwT
%
%Analogue of Theorem 1.8 is stated as follows: 
%\begin{tcolorbox}[enhanced,attach boxed title to top center={yshift=-3mm,yshifttext=-1mm},
%  colback=blue!5!white,colframe=blue!75!black,colbacktitle=red!80!black,
%  title=Theorem 1.8's Analogue,fonttitle=\bfseries,
%  boxed title style={size=small,colframe=red!50!black} ]
%Let $u_1, u_2, \hdots , u_n$  be real measurable functions on a measurable space $X$, let $f$ be a continuous mapping of $\mathbb{R}^{n}$ into a topological space $Y$, and define:
%\begin{equation*}
%    h(x) = f(u_1(x), u_2(x), \hdots , u_n(x))
%\end{equation*}
%for $x \in X$. Then $h: X \rightarrow Y$ is measurable.
%\end{tcolorbox}
\section{Notation}
\label{s:notation}
\begin{itemize}[label =  $\circ$, leftmargin = *]
\item $M_n(\R) \coloneqq$ set of $n \times n$ matrices with real entries
\item $GL_n(\R) \Let$ set of $n \times n$ invertible matrices with real entries
\item $O(n) \Let \{Q \in GL_n(\R) \suchthat QQ^{t} = \I_n\}$ is called the set of $n \times n$ real, orthogonal matrices
\item $SO(n) \Let \{Q \in O(n) \suchthat \mathrm{det}(Q) = 1\}$ is called the set of special orthogonal matrices
\end{itemize}

\section{Preliminaries}
\label{s:prelims}
\subsection{The Orthogonal Matrices as a \emph{Group}}
\begin{theorem}
$O(n)$ forms a group under the operation of matrix multiplication.
\end{theorem}
\emph{Proof}: Consider $Q, P \in O(n)$. Then,
\[
QP(QP)^{t} = QPP^{t}Q^{t} = \I_n \quad \text{(closure under multiplication),}
\]
\[
Q\I_n = Q = \I_n Q \quad \text{($\I_n$ is the identity element),}
\]
\[
QQ^{t} = \I_n \quad \text{($Q^{t}$ forms the inverse).}
\]
We notice that $O(n)$ forms a subgroup of $GL_n(\R)$. \\
\emph{Remark.} For all $Q \in O(n)$, we have that $\mathrm{det}(Q) = \pm 1.$ \emph{Proof}: $\mathrm{det}(\I_n) = 1 = \mathrm{det}(QQ^{t}) = \bigl(\mathrm{det}(Q)\bigr)^{2} \implies \mathrm{det}(Q) = \pm 1.$

\begin{theorem}
$SO(n)$ forms a normal subgroup of $O(n)$.
\end{theorem}
\emph{Proof}: Consider $Q \in O(n)$ and $P \in SO(n)$. Then, $QPQ^{-1} \in O(n)$ since $Q, P, Q^{-1} \in O(n)$ and $O(n)$ is closed under multiplication. Moreover,
\[
\mathrm{det}(QPQ^{-1}) = \mathrm{det}(Q)\mathrm{det}(P)\mathrm{det}(Q^{-1}) = \mathrm{det}(P) = 1.
\] 
Thus, $QPQ^{-1} \in SO(n)$ $\forall$ $Q \in O(n) \implies SO(n) \lhd O(n) \preccurlyeq GL_n(\R)$.

\subsection{The Orthogonal Group with a \emph{Topology}}
We can use the Euclidean metric on $\R[n^2]$ to assign a metric to  $M_n(\R)$.
\begin{theorem}
Let $A, B \in M_n(\R)$. Then,
\[
d: M_n(\R) \mapsto M_n(\R); \quad d(A, B) \Let \sqrt{\sum_{1\leq i, j \leq n}\bigl(A_{ij} - B_{ij}\bigr)^{2}},
\]
is a metric on $M_n(\R)$.
\end{theorem}
\emph{Proof}: We check the 3 properties a metric $d$ must follow: \\
$d(A, B) = 0 \implies A_{ij} = B_{ij}$ $\forall$ $1 \leq i, j \leq n \implies A = B$.

$d(A, B) = \sqrt{\sum_{1\leq i, j \leq n}\bigl(A_{ij} - B_{ij}\bigr)^{2}} = d(B, A)$.

\[
d(A, B) = \sqrt{\sum_{1\leq i, j \leq n}\bigl(A_{ij} - B_{ij}\bigr)^{2}} = \sqrt{\sum_{1\leq i, j \leq n}\bigl(A_{ij} - C_{ij} + C_{ij} B_{ij}\bigr)^{2}}  
\]
\[
\leq \sqrt{\sum_{1\leq i, j \leq n}\bigl(A_{ij} - C_{ij}\bigr)^{2}} + \sqrt{\sum_{1\leq i, j \leq n}\bigl(C_{ij} - B_{ij}\bigr)^{2}} = d(A,C) + d(C,B) 
\]
Hence $d$ defined as above forms a metric on $M_n(\R)$. It is important to note that we have induced the notion of a metric by viewing the entries of $M_n(\R)$ in the Euclidean space $\R[n^2]$ and using the standard Euclidean metric.

\section{The Orthogonal Group as a Topological Group}
\label{s:results}
\subsection{Topological Group}
\begin{theorem}
 The matrix group $O(n)$ is a topological group.
\end{theorem}
\emph{Proof}: To prove this, we must show that matrix multiplication and inverses are both continuous operations on $M_n(\R)$. However, we note that matrix multiplication can be expressed as a polynomial in the entries. Since polynomials are continuous functions, we get that multiplication of matrices is element-wise continuous and thus continuous on the whole matrix. The inverse of a matrix $A \in M_n(\R)$ is given by $A^{-1} \Let \frac{\mathrm{adj}(A)}{\mathrm{det}(A)}$. Since the adjoint and determinant are both polynomials on the entries of $A$, they are continuous functions (and the determinant is non-zero for invertible matrices). Hence, we get that the inverse is also a continuous map.

\subsection{Compactness of O(n)}
\begin{lemma}
$O(n)$ is a closed subset of $\R[n^2]$.
\end{lemma}
\emph{Proof}: Consider $f: M_n(\R) \mapsto M_n(\R)$ defined as $f(Q) = QQ^{t}$. Then $\bigl(f(Q)\bigr)_{ij} = \sum_{k=1}^{n}Q_{ik}Q_{jk}$ which is a polynomial in the entries of $Q$ and polynomials are continuous functions. Since $f$ is element wise continuous, we conclude $f$ is a continuous map on $M_n(\R)$. Now, $O(n) = f^{-1}(\{\I_n\})$ which is the inverse of a closed set. Hence, $O(n)$ is closed.
\begin{lemma}
$O(n)$ is a closed subset of $\R[n^2]$.
\end{lemma}
\emph{Proof}: $O(n) \subseteq \{A \in M_n(\R) \suchthat \sum_{i=1}^{n} A_{ij}^{2} = 1\}$ $\forall$ $1 \leq j \leq n$. Clearly this set is bounded by the open ball given by $d(A, 0) = \sum_{1 \leq i,j \leq n}A_{ij}^{2} = n^2$ where $0$ is the zero matrix with all entries 0.

\begin{theorem}
$O(n)$ is compact.
\end{theorem}
\emph{Proof}: As an application of the Heine-Borel theorem, since $O(n)$ is a closed and bounded subset
of Euclidean $\R[n^2]$, we have that O(n) is compact.

\subsection{Path-Connectedness}
\begin{theorem}
$O(n)$ is disconnected.
\end{theorem}
\emph{Proof}: Note that the function $\mathrm{det}: M_n(\R) \mapsto \R$ is a continuous function. Now, $\mathrm{det}: O(n) \mapsto \{-1, 1\} \subset \R$. Since continuous functions map connected sets to connected sets, we have that $O(n)$ is disconnected.
\begin{theorem}
$SO(n)$ is path-connected.
\end{theorem}
\emph{Proof}: We begin by inspecting the simple case of $SO(2)$ and build our way upwards. Any generic matrix $P \in SO(2)$ is given by $P = \begin{pmatrix}
\cos(\theta) & -\sin(\theta) \\
\sin(\theta) & \cos(\theta)
\end{pmatrix}$. To show $SO(2)$ is path-connected, it is sufficient to show there exists a path between any arbitrary element in $SO(2)$ and the identity matrix. Consider the function $\gamma: [0,1] \mapsto SO(2)$ defined as
$$\gamma(t) = \begin{pmatrix}
\cos(t\theta) & -\sin(t\theta) \\
\sin(t\theta) & \cos(t\theta)
\end{pmatrix}.$$
Then observe that $\forall$ $t \in [0,1]$, $\gamma(t) \in SO(2)$ and $\gamma(0) = \I_2$ while $\gamma(1) = P$. Thus, we have shown $SO(2)$ is path-connected. Similarly, to show $SO(3)$ is path connected, consider $P \in SO(3).$ Since $SO(3) \lhd O(3) \implies \exists$ $Q \in O(3)$ and $P'_{\theta} = \begin{pmatrix}
1 & 0 & 0 \\
0 & \cos(\theta) & -\sin(\theta) \\
0 & \sin(\theta) & \cos(\theta)
\end{pmatrix} \in SO(3)$ such that, 
$$P = QP'Q^{-1} = QP'_{\theta}Q^{t}.$$ Thus, consider the path $\gamma(t) = QP'_{t\theta}Q^{t}$. Then, $\gamma(0) = \I_3$ and $\gamma(1) = P$. Thus, we have shown $SO(3)$ is path connected. More generally, for any $P \in SO(n)$, there exists $Q \in O(n)$ and $P'_{\theta} \in SO(n)$ given by $P'_{\theta} = \begin{pmatrix}
R_{\theta_1} \\
& R_{\theta_2} \\
& & R_{\theta_3} \\
& & & \ddots \\ 
& & & & R_{\theta_m}\\ 
\end{pmatrix}$ such that $P = QP'Q^{t}$. Every $R_{\theta_k}$ is either a $2\times 2$ rotation matrix or equal to $[1]$. Then the path connecting $P$ to $\I_n$ is given by $\gamma(t) = QP'_{t\theta}Q^{t}$ where 
 $\gamma(0) = \I_n$ and $\gamma(1) = P$. Thus we have shown that $SO(n)$ is path connected. 
\begin{theorem}
$O(n)\setminus SO(n)$ is path-connected.
\end{theorem}
\emph{Proof}: We will show there exists a path between any two elements $X, Y \in O(n)\setminus SO(n)$. Consider $Q \in O(n)\setminus SO(n)$. Then, $QX, QY \in SO(n)$ since $QX, QY \in O(n)$ $\bigl(O(n)$ is closed under multiplication$\bigr)$. Moreover $\mathrm{det}(QX) = \mathrm{det}(Q)\mathrm{det}(X) = -1\times-1 = 1 = \mathrm{det}(QY) \implies QX, QY \in SO(n)$. Using the above result, we know $SO(n)$ is path-connected implying there exists a path connecting $QX$ and $QY$ called $\gamma(t)$ such that $\gamma(0) = QX$ and $\gamma(1) = QY$. Then the path $\phi(t) = Q^{-1}\gamma(t)$ is a well defined continuous function since matrix multiplication is continuous (according to Theorem 3.1). Clearly $\phi(0) = Q^{-1}\gamma(0) = Q^{-1}QX = X$ and similarly $\phi(1) = Y$. Thus we have shown there exists between any two elements $X, Y \in O(n) \setminus SO(n)$ which implies it is a connected set. \\
\emph{\textbf{Remark}}: It follows that $O(n)$ has exactly two path components. 
\section{Gram-Schmidt as a Deformation Retract}
\label{s:deformation_retract}
\begin{definition}[Gram-Schmidt orthogonalization]
The Gram-Schmidt orthogonalization ($GS$) of a matrix $A$ of the form $A \Let (v_1, v_2, \ldots , v_n)$ where $v_i \in \R[n]$ are the column vectors of $A$ is given by,
$GS: GL_n(\R) \times [0, \frac{1}{2}] \mapsto GL_n(\R)$ is defined as,
$$GS(A, t) \Let \bigl((1-2t)v_1 + 2tu_1, \ldots, (1-2t)v_n + 2tu_n\bigr),$$ where each $u_i$ is defined as
$$u_1 = v_1,$$
$$u_2 = v_2 - \frac{\innerprod{u_1, v_2}}{\norm{u_1}}u_1,$$
$$u_n = v_n - \sum_{k=1}^{n-1}\frac{\innerprod{u_k, v_n}}{\norm{u_k}}u_k.$$
\end{definition}
Since Gram-Schmidt orthogonalization generates an orthogonal set of $n$ vectors from a linearly independent set of $n$ vectors, the image of $A$ is equal to image of $GS(A, t)$ implying $GS(A, t)$ attains full rank for all $t$. Hence for all $t \in [0, \frac{1}{2}]$, $GS(A,t) \in GL_n(\R).$ 
\begin{definition}[Normalization map]
The normalization map for a matrix $A$ as defined above, given by $N: GL_n(\R) \times [\frac{1}{2}, 1] \mapsto GL_n(\R)$ defined by:
$$N(A,t) \Let \bigl(v_1(\frac{2t-1}{\norm{v_1}} + 2 - 2t), \ldots , v_n(\frac{2t-1}{\norm{v_n}} + 2 - 2t)\bigr).$$
\end{definition}
The map $N$ is designed such that $N(A,\frac{1}{2}) = A$ and $N(A, 1) = \bigl(\frac{v1}{\norm{v1}}, \ldots ,\frac{v_n}{\norm{v_n}}\bigr)$. We must check that for all $t \in [\frac{1}{2},1]$, $N(A,t) \in GL_n(\R)$. Computing the determinant, given by
$$\mathrm{det}\bigl(N(A,t)\bigr) = \mathrm{det}(A)\prod_{i=1}^{n}(\frac{2t-1}{\norm{v_i}} + 2 - 2t).$$
We must check that $\frac{2t-1}{\norm{v_i}} + 2 - 2t \neq 0$ $\forall$ $1\leq i \leq n$. Assume this is true for some $i$. Then,
$$\frac{2t-1}{\norm{v_i}} + 2 - 2t = 0$$
$$\implies 2t - 1 = 2t\norm{v_i} - 2\norm{v_i}$$
$$\implies t = \frac{1 - 2\norm{v_i}}{2(1 - \norm{v_i})}.$$
Plotting the function $y = \frac{1-2x}{2(1-x)}$on a graphing-calculator (Desmos) shows that it has no solutions for $y \in [\frac{1}{2}, 1]$. Hence, we have shown a contradiction which implies that $\mathrm{det}\bigl(N(A,t)\bigr) \neq 0 \implies N(A,t) \in GL_n(\R)$ $\forall$ $t \in [\frac{1}{2},1].$ 

\begin{definition}[Gram-Schmidt orthonormalization]
Defining the Gram-Schmidt ortho-normalization (GSO), given by $GSO: GL_n(\R) \times [0,1] \mapsto GL_n(\R)$ and defined by
\[
    GSO(A,t) = 
\begin{cases}
    GS(A,t),& \text{if } t\in [0, \frac{1}{2})\\
    N(A,t),& \text{if } t\in [\frac{1}{2}, 1]
\end{cases}.
\]
This is well-defined since $GS(A,t)$ and $N(A,t)$ agree at $t=\frac{1}{2}$.
\end{definition}

\begin{theorem}
The $GSO$ as defined above is a strong deformation retraction of $GL_n(\R)$ into $O(n)$.
\end{theorem}
\emph{Proof}: Observe that $GSO(A,0) = A \in GL_n(\R).$ Next, $GSO(A,1)$ is a matrix with ortho-normal columns implying $GSO(A,1) \in O(n)$. Thus, $$

\end{document}
